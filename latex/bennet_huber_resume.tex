% LaTeX file for resume 
% This file uses the resume document class (res.cls)

\documentclass{res} 
\usepackage{helvetica} % uses helvetica postscript font (download helvetica.sty)
\usepackage{newcent}   % uses new century schoolbook postscript font 
\usepackage{verbatim}
\usepackage[letterpaper,left=0.6in,right=1.1in,top=0.5in,bottom=0.5in]{geometry}
\usepackage[T1]{fontenc}  % https://tex.stackexchange.com/questions/312/correctly-typesetting-a-tilde/377#377
%\usepackage{titlesec}
%\setlength{\textheight}{10in} % increase text height to fit on 1-page


%\titlespacing\section{2pcs}{12pt plus 4pt minus 2pt}{0pt plus 2pt minus 2pt}

\begin{document} 

\name{Bennet Huber}     % the \\[12pt] adds a blank
				        % line after name      

\vspace{-24pt}
\address{bennet.huber@gmail.com\\(215) 490-4297}
\address{1219 NW Richmond Beach Rd\\Shoreline, WA 98177}

\begin{resume}

\section{Job Objective}
    To work as a lead engineer helping my team solve challenging and exciting problems.
 
\section{Education}
    The Pennsylvania State University - May 2010\\
    Bachelors of Science in Computer Science with a minor in Mathematics\\
    GPA 3.55, Dean's list 7/8 semesters
 
\section{Experience}
   \vspace{-0.1in}
   \begin{tabbing}%
   \hspace{2.2in}\= \hspace{2.2in}\= \kill % set up two tab positions
   {\bf Senior SDE}  \> Amazon Inc\> April 2021 - Present\\
   {\bf SDE II}          \> Seattle, WA \> April 2016 - April 2021
   \end{tabbing}\vspace{-17pt}
   {\large \bf General Amazon Stats}
   \begin{itemize}
   \item Code: \textasciitilde 1900 commits, \textasciitilde 2000 wiki edits, \textasciitilde 380 Code Reviews (CRs) submitted, \textasciitilde 1000 CRs read
   \item \textasciitilde 1000 tickets resolved
   \item Mentored \textasciitilde 14 junior engineers/interns, 4 of whom are now senior engineers at Amazon
   \item Conducted \textasciitilde 150 interviews
   \item Amazon has an internal stackoverflow.com clone called Sage.  For Sage points, I'm ranked in the top 150 in the    company, or top 0.3\% of active users.  Most of my participation has been answering customer questions about Datapath.
   \item Participated in oncall rotations for two critical services (FMA, Datapath Responders)
   \end{itemize}
   
   {\large \bf Datapath Platform }\\
   Datapath is a proprietary serverless execution platform and programming language designed for high throughput, low latency execution of business logic organized in an SOA architecture.  It powers a significant portion of the real-time backend business logic for amazon.com traffic, and is used by hundreds of internal teams and thousands of developers.  It's a little bit like several AWS services rolled into one cohesive platform - it doesn't directly own any business logic; it's a platform for other teams to use.\\
   \\
   {\bf Datapath Responders (April 2016 - August 2016) }\\
   The Responders team at the time owned the language, execution logic, and real-time runtime fleets for executing customer logic.  I was officially on Responders only a short time before being reorged, which I mostly spent learning the system and codebase.  I was in oncall rotations from June 2016 - January 2017, and participated in several full website outage incidents.  My main project was building a tool to allow customers to find which top level REST bindings could transitively invoke their business logic.  The core codebase was later adapted to solve several other similar analysis problems.\\
   \\
   {\bf Datapath LTCX (August 2016 - February 2019) }\\
   In summer 2016 the Datapath platform was approved to double their headcount; a team reorganization was necessary.  Due to my personal interests, I was the first (and initially only) member of the newly formed Language, Tools, and Customer Experience (LTCX) team, responsible mainly for the developer experience of platform users.  This included the build tools, permissions system, deployment system, billing system, and internal web portals to expose all this functionality to developers.  My main accomplishments included:
   \begin{itemize}
   \item Setting team priorities from the outset.  As a former user of the product, I had insight into what problems our platform faced, and as the only member of the team I was left to my own devices on what to prioritize and work on.
	\item With a small amount of help from a senior engineer on another team, I successfully trained and onboarded six new team members in the course of the first six months, including our new manager.  Of the six developers on that team, four have since been promoted to Senior SDEs at Amazon (including myself).
   \item As part of a data migration project, I added the capability for the entire platform to resolve customer business logic from non-Prod data sources.  Before this, many features could only be tested by publishing user code to production, making some changes simply impossible to test safely.  This required auditing, modifying, and testing access to the data store across three critical services, dozens of packages, and hundreds of thousands of lines of Java code.
   \item In early 2017, builds for the shared packages used throughout Datapath had grown to be four hours long, with two hours accounted for in a single top-level package my team owned.  I identified this as a major impediment to developer progress throughout the organization, as every Code Review and every deployment must do a full build of all consuming packages.  I identified the slow unit tests, developed several optimization techniques, and worked with the owning teams to apply them to the relevant code packages.  The combined optimizations reduced our package's test runtime from 2 hours to 4.5 minutes, and the full build time for the shared packages from 4 hours to under 30 minutes.
   \item Developed a novel testing framework for testing equals()/hashcode() consistency across a complex type hierarchy.  Changes to our core data model had introduced many bugs of this type over several years, resulting in numerous failed customer deployments and several outages for amazon.com.  My test framework automatically identified all existing bugs, and ensured bugs introduced by new additions/modifications would be caught in the future.
   \item The Datapath Deployer is responsible for ingesting customer code changes and sending updates to all relevant server fleets to run the new code versions, with no downtime.  If the Deployer doesn't work, customers can't update the business logic for their services that power amazon.com's backend.  The Deployer had slowly evolved from a simple shell script to a complex, convoluted, and tightly coupled Java codebase with lots of implicit and shared mutable state.  As a result, changes to the internal business logic would often break when deployed to production, despite good test coverage of individual components.\\
   I successfully refactored the internal Deployer code, breaking the monolithic parts into decoupled pieces and removing much of the shared mutable state.  This took several weeks of effort.  Future changes were made with much fewer problems manifesting outside of unit/integration tests.
   \item Amazon does not have an official position of technical lead, but I filled that role over the lifetime of LTCX
   \end{itemize}
   {\bf Datapath Artifacts Team (February 2019 - April 2021) }\\
   In February 2019 Datapath teams were again reorganized, and I ended up on the newly formed Artifacts team, which owns the global language data store and deployment system that updates both it and the runtime fleets when customers deploy new versions of their code.  As one of the senior developers on Artifacts, I spend a lot of time training and assisting newer team members in addition to my own work.  Some of my accomplishments while on the Artifacts team included:
   \begin{itemize}
   \item In early 2019 we had significant new work planned for the Deployer system.  Over the course of six weeks I developed a long term architectural vision document for a system that could successfully deliver our long term feature goals.  This vision has guided much of my team's major project goals ever since.  From a high level, it takes existing business logic implemented in two monolithic systems and splits them out into a microservice architecture.  One of the microservices has already been fully developed, two others have begun development this year.
   \item Partnered with a teammate to build a custom telemetry tracking service for our Deployment system (from the architecture vision above), allowing customers to track their deployments through our systems in greater detail.  The service is built entirely in native AWS and uses Lambda, RDS, and API Gateway.
   \item Successfully drove a campaign to eliminate failed customer deployments due to transient system errors as much as possible.  These were caused by several subtle concurrency issues present in our distributed deployment system, some of which were several years old.  I also lead customer communication on the issue.
   \item I identified several inefficiencies in the Datapath language build and test runners, leading to a 3x - 10x build time improvement for all our customers.  One of the solutions I developed was later leveraged to provide similar speedups to a separate customer test tool.
   \item Traditional Service Oriented Architectures (SOAs) are normally constructed of sets of loosely coupled microservices that call each other in a dependency graph.  These services are deployed separately to their own independent fleets.  As a service owner, a common testing methodology is to deploy new updates to a single host (or small set of hosts) in the service fleet and monitor it for problems for some amount of time, before fully deploying to the rest of the fleet.  This technique is known as "onebox testing", and is a last line of defense to catch any problems before they're fully deployed to production.\\
At the end of 2019, onebox test support was a top feature request from our biggest customers.  Unfortunately, Datapath is not a traditional SOA - rather than deploying each "service" to separate fleets, each is deployed to every host across a single fleet.  This makes even defining what "onebox testing" means difficult in Datapath.  I owned the Customer Onebox project from start to delivery, including gathering requirements from internal customers, design, splitting the project into tasks, assigning those tasks to junior engineers on my team (and of course myself), and partnering with our fleet management team to deploy the new onebox testing fleets.  The first beta customers were onboarded at the end of 2020.  The project is still in beta testing.
   \item Extensively mentored 5 new hires
   \item Promoted to Senior Software Development Engineer (SDE III) 4/2021
   \end{itemize}

   \begin{tabbing}%
   \hspace{2.2in}\= \hspace{2.2in}\= \kill % set up two tab positions
   {\bf SDE I}  \> Amazon Inc\> August 2014 - April 2016\\
                          \> Seattle, WA
   \end{tabbing}\vspace{-17pt}
   {\large \bf Featured Merchant Algorithm (FMA) Team}\\
   FMA owns the logic that picks which merchant offer is tied to Amazon's "Add to Cart" button. It is a low latency, high throughput system - its logic is invoked millions of times per second worldwide with latency on the order of 10s of milliseconds.  It is built on the Datapath platform (see other Amazon experience).  While on FMA, some of my accomplishments included:
   \begin{itemize}
   \item Developed a shadow traffic automated testing mechanism to reduce developer time to run integration tests from several hours to a few minutes.
   \item Designed and implemented a new version of FMA's "public" API vended to other Amazon teams.  It was used for \textasciitilde 3 years before being deprecated and replaced by an upgraded version to incorporate new product features (original is still in use in production).
   \item Improved stability of ETL system for gathering data analytics - went from regularly failing without notification (resulting in data loss) to full alarming and advanced retry logic
   \item Added latency-neutral incorporation of competitive pricing to FMA algorithm and metrics gathering through downstream services under significant time pressure (6 weeks).
   \item Promoted to SDE II November 2015
   \end{itemize}
   
   
   \begin{tabbing}
   \hspace{2.2in}\= \hspace{2.2in}\= \kill % set up two tab positions
   % \hspace{2.3in}\= \hspace{2.6in}\= \kill % set up two tab positions
    {\bf Software Developer} \>Azavea Inc     \>December 2010 - April 2014\\
                             \>Philadelphia, PA
   \end{tabbing}\vspace{-17pt}      % suppress blank line after tabbing
   Azavea is a medium sized consulting firm specializing in web/mobile applications related to GIS problems.  I worked on teams of two to six people and projects lasting from several months to many years.  I was a full-stack developer working on things such as UI design and implementation, application logic, spatial predictive modeling, distributed architecture design, mobile app development, application deployment, and database schema design.  One of the primary projects I worked on is called HunchLab, a product designed for crime analysts that implements many cutting-edge crime prediction algorithms and statistical models.
   \begin{tabbing}%
   \hspace{2.2in}\= \hspace{2.2in}\= \kill % set up two tab positions
   {\bf Intern}  \> Cisco Systems\> Summers 2007 and 2008\\
                          \> San Jose, CA
   \end{tabbing}\vspace{-17pt}
   Internal web development and server administration.


\section{Relevant Skills}
Proficient in Java development, designing and maintaining highly available distributed systems, microservices/SOA architectures, and mentoring junior engineers.  I've worked on a large variety of platforms and technologies, including C\#/.NET, Python, AWS, Javascript, IOS/Android, and too many others to list.
 
\section{Activities}
    Recipient of the Lockheed Martin Engineering Scholars Award\\
    Child Rearing\\
    Past instructor for Girl Develop It Philly\\
    Too many local hackathons to enumerate\\
    Bicycled across America, Summer 2009

\section{References}
    Available upon request
 
\end{resume}
\end{document}
