% LaTeX file for resume 
% This file uses the resume document class (res.cls)

\documentclass{res} 
\usepackage{helvetica} % uses helvetica postscript font (download helvetica.sty)
\usepackage{newcent}   % uses new century schoolbook postscript font 
\usepackage{verbatim}
% Add "showframe" to args below to see outline
\usepackage[letterpaper,left=0.6in,right=1.1in,top=0.5in,bottom=0.5in]{geometry}
\usepackage[T1]{fontenc}  % https://tex.stackexchange.com/questions/312/correctly-typesetting-a-tilde/377#377
\usepackage{array}
\usepackage{tabularx}
\usepackage{printlen}
%\usepackage{titlesec}
\setlength{\textheight}{10in} % increase text height to fit on 1-page

%\titleformat{\section}
%  {\normalfont\Large\bfseries}{\thesection}{1em}{}
%\titlespacing\section{2pcs}{12pt plus 4pt minus 2pt}{0pt plus 2pt minus 2pt}

\newlength{\vsep}
\setlength{\vsep}{\baselineskip}


\begin{document} 

\name{Bennet Huber}     % the \\[12pt] adds a blank
				        % line after name      

\vspace{-24pt}
\address{bennet.huber@gmail.com\\(215) 490-4297}
\address{1219 NW Richmond Beach Rd\\Shoreline, WA 98177}

\begin{resume}

\section{Job Objective}
    To utilize my extensive software development experience in a flexible part-time work environment.
 
\section{Education}
    The Pennsylvania State University - May 2010\\
    Bachelors of Science in Computer Science with a minor in Mathematics\\
    GPA 3.55, Dean's list 7/8 semesters

\section{Experience}
   \begin{tabularx}{\textwidth}{@{}>{\bf}l>{\large\bf\centering\arraybackslash}Xr@{}}
   Chief Software Architect & Compose AI -- Remote & August 2021 - November 2022\vspace{\vsep}\\
   \end{tabularx}\\
   Compose AI was an early mover startup in the AI text generation space.  Their primary product is a chrome extension written in ReactJS that allows users to access various AI text capabilities in the web browser, such as autocompletion and ChatGPT prompted text generation.  It was very small while I was there, only three FTEs and a rotating cast of contractors.
   \begin{itemize}
   \item Built test bench to automate accuracy testing of autocomplete feature
   \item Refactored backend to scale to multiple model servers for autocomplete requests, with load shedding to degrade gracefully under high load
   \item Built "Compose Now" UI feature, which allows users to invoke ChatGPT from a prompt and automatically insert the results into any text box in the browser
   \item Built a prototype MacOS desktop version of the app using Hammerspoon and Electron in two weeks
   \item Mentored and interviewed junior engineers
   \end{itemize}
   \vspace{\vsep}

   %%%%%%%%%%%%%%%%%%%%%%%%%%
   \begin{tabularx}{\textwidth}{@{}>{\bf}l>{\large\bf\centering\arraybackslash}Xr@{}}
   Senior SDE & Amazon Inc -- Seattle, WA & April 2021 - August 2021\\
   SDE II & & November 2015 - April 2021\\
   SDE I & & August 2014 - November 2015\vspace{\vsep}\\
   \end{tabularx}\\
   {\large \bf General Amazon Stats}
   \begin{itemize}
   \item Code: \textasciitilde 1900 commits, \textasciitilde 2000 wiki edits, \textasciitilde 380 Code Reviews (CRs) submitted, \textasciitilde 1000 CRs read
   \item \textasciitilde 1000 tickets resolved
   \item Mentored \textasciitilde 14 junior engineers/interns, 4 of whom were senior engineers at Amazon when I left
   \item Conducted \textasciitilde 150 interviews
   \item Participated in oncall rotations for two critical services (FMA, Datapath Responders)
   \end{itemize}
   
   {\large \bf Datapath Platform }\\
   Datapath is a proprietary serverless execution platform and programming language designed for high throughput, low latency execution of business logic organized in an SOA architecture.  It powers a significant portion of the real-time backend business logic for amazon.com traffic, and is used by hundreds of internal teams and thousands of developers.  It's a little bit like several AWS services rolled into one cohesive platform - it doesn't directly own any business logic; it's a platform for other teams to use.\\
   \\
   {\bf Datapath Artifacts Team (February 2019 - August 2021) }\\
   I joined the Artifacts team as part of a reorg; the team owned the global deployment system for all Datapath code.
   \begin{itemize}
   \item Developed a long term architectural vision document for a deployment system that could successfully deliver our long term feature goals.  This vision guided much of my team's major project goals for several years until I left.
   \item Partnered with a teammate to build a custom telemetry tracking service for our Deployment system (from the architecture vision above), allowing customers to track their deployments through our systems in greater detail.  The service was built entirely in native AWS and uses Lambda, RDS, and API Gateway.
   \item Successfully drove a campaign to eliminate failed customer deployments due to transient system errors as much as possible.  These were caused by several subtle concurrency issues present in our distributed deployment system, some of which were several years old.  I also led customer communication on the issue.
   \item Improved Datapath customer build times by 3x - 10x.
   \item Lead a project to add onebox test functionality to our deployments.  I owned the Customer Onebox project from start to delivery, including gathering requirements from internal customers, design, splitting the project into tasks, assigning those tasks to junior engineers on my team (and of course myself), and partnering with our fleet management team to deploy the new onebox testing fleets.  The project took several man-years of effort, and was still in beta testing when I left.
   \item Extensively mentored 5 new hires; several were promoted
   \item Promoted to Senior Software Development Engineer (SDE III) 4/2021
   \end{itemize}
   {\bf Datapath LTCX (August 2016 - February 2019) }\\
   As part of a reorg, I was assigned as the first, and initially only, member of the newly formed Language, Tools, and Customer Experience (LTCX) team, responsible mainly for the developer experience of platform users.  This included the build tools, permissions system, deployment system, billing system, and internal web portals to expose all this functionality to developers.  My main accomplishments included:
   \begin{itemize}
   \item Built the team - with a small amount of help from a senior engineer on another team, I successfully trained and onboarded six new team members in the course of the first six months, including our new manager.
   \item As part of a data migration project, I added the capability for the entire platform to resolve customer business logic from non-Prod data sources.  Before this, many features could only be tested by publishing user code to production, making some changes simply impossible to test safely.  This required auditing, modifying, and testing access to the data store across three critical services, dozens of packages, and hundreds of thousands of lines of Java code.
   \item Improved full build + test times across the org from >4 hours to <30 minutes.
   \item Developed a novel testing framework for testing equals()/hashcode() consistency across a complex type hierarchy.
   \item Amazon does not have an official position of technical lead, but I filled that role over the lifetime of LTCX
   \end{itemize}
   {\bf Datapath Responders (April 2016 - August 2016) }\\
   The Responders team owned the language, execution logic, and real-time runtime fleets for executing customer logic.
   \begin{itemize}
   \item Participated in oncall rotation from June 2016 - January 2017, helped with several full website outage incidents
   \item Designed and built code dependency analysis tool for users
   \end{itemize}
   \vspace{\vsep}
   {\large \bf Featured Merchant Algorithm (FMA) Team (August 2014 - April 2016)}\\
   FMA owns the logic that picks which merchant offer is tied to Amazon's "Add to Cart" button. It is a low latency, high throughput system - its logic is invoked millions of times per second worldwide with latency on the order of 10s of milliseconds.  It is built on the Datapath platform (see other Amazon experience).  While on FMA, some of my accomplishments included:
   \begin{itemize}
   \item Developed a shadow traffic automated testing mechanism to reduce developer time spent running integration tests from several hours to a few minutes.
   \item Designed and implemented a new version of FMA's "public" API vended to other Amazon teams.  It was used for \textasciitilde 3 years before being deprecated and replaced by an upgraded version to incorporate new product features (original is still in use in production).
   \item Improved stability of ETL system for gathering data analytics - went from regularly failing without notification (resulting in data loss) to full alarming and advanced retry logic
   \item Added latency-neutral incorporation of competitive pricing to FMA algorithm and metrics gathering through downstream services under significant time pressure (6 weeks).
   \item Promoted to SDE II November 2015
   \end{itemize}
   \vspace{\vsep}
   
   \begin{tabularx}{\textwidth}{@{}>{\bf}l>{\large\bf\centering\arraybackslash}Xr@{}}
   Software Developer & Azavea Inc -- Philadelphia, PA & December 2010 - April 2014\vspace{\vsep}\\
   \end{tabularx}\\
   Azavea was a medium sized consulting firm specializing in web/mobile applications related to GIS problems.  I worked on teams of two to six people and projects lasting from several months to many years.  I was a full-stack developer working on things such as UI design and implementation, application logic, spatial predictive modeling, distributed architecture design, mobile app development, application deployment, and database schema design.  One of the primary projects I worked on is called HunchLab, a product designed for crime analysts that implements many cutting-edge crime prediction algorithms and statistical models.\\

   \begin{tabularx}{\textwidth}{@{}>{\bf}l>{\large\bf\centering\arraybackslash}Xr@{}}
   Intern & Cisco Systems -- San Jose, CA & Summers 2007 and 2008\vspace{\vsep}\\
   \end{tabularx}\\
   Internal web development and server administration.


\section{Relevant Skills}
Proficient in Java development, designing and maintaining highly available distributed systems, microservices/SOA architectures, writing, and mentoring junior engineers.  I've worked on a large variety of platforms and technologies, including C\#/.NET, Python, AWS, Typescript/Javascript, IOS/Android, and too many others to list.
 
\section{Activities}
    Father of two small children\\
    Past instructor for Girl Develop It Philly\\
    Too many local hackathons to enumerate\\
    Bicycled across America, Summer 2009
    Recipient of the Lockheed Martin Engineering Scholars Award\\

\section{References}
    Available upon request

\end{resume}
\end{document}
